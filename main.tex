\documentclass{beamer}
\usetheme{Copenhagen}
\usecolortheme{default}

\usepackage{stmaryrd}
\usepackage{amsmath, amssymb}
\usepackage{tikz}

\setbeamertemplate{headline}{}
%Information to be included in the title page:
\title{A Variety of Belief Update Operators}
\author{Omar Gamal 52-9805 \\Abdullah Elnahas 52-4957}
\institute{Supervised by: \textbf{Prof. Haythem O. Ismail}}
\date{May 14, 2025}

\begin{document}

\frame{\titlepage}

\section{Introduction}

\section{Language and Preliminaries}
\subsection{Syntax and Atoms}

\section{Update Operators}
\subsection{PMA, FORBUS, MCD, MCD*, WSS, WSS↓, WSS$^{\text{dep}}$, WSS$\downarrow^{\text{dep}}$, MCE, and $MPMA^{\gg}$}

\section{Strength Comparison}

\section{KM Postulates Evaluation}
\subsection{Postulate Satisfaction}
\subsection{Violations and Justifications}

\subsection{Failure of PMA}
\subsection{Better Handling in MCD, MCE}

\section{Integrity Constraints}

\section{Other Approaches}

\section{Conclusion}

\begin{frame}
\frametitle{Outline}
\small
\tableofcontents
\end{frame}

%=========INTRO=========%

\begin{frame}
\frametitle{Introduction: Why Belief Update?}
\begin{itemize}
    \item Belief bases = sets of propositional formulas an agent believes.
    \item The world can change, then beliefs must adapt.
    \item Goal: update beliefs with minimal change.
\end{itemize}
\end{frame}

%=========LANGUAGE AND PRELIMINARIES=========%

\begin{frame}
\frametitle{Propositional Language}
\begin{itemize}
    \item The language is built from a set of atomic propositions: $\text{ATM} = \{p, q, r, \ldots\}$.
    \item Formulas are constructed using:
    \[
        \neg, \quad \wedge, \quad \vee, \quad \rightarrow, \quad \leftrightarrow, \quad \bot, \quad \top
    \]
    \item A \textbf{literal} is an atom or its negation: $p$, $\neg p$
    \item A \textbf{clause} is a disjunction of literals: $p \vee \neg q \vee r$
    \item We use $\mathit{atm}$ to refer to the set of atoms used in a specific formula.
\end{itemize}
\end{frame}

\begin{frame}
\frametitle{Redundant Atoms}
\begin{itemize}
    \item An atom $p$ is \textbf{redundant in} $A$ if changing its value does not affect the truth of $A$.
    \item Formally:
    \[
    A[p / \top] \leftrightarrow A[p / \bot]
    \]
    \item This allows simplification of formulas and belief bases.
\end{itemize}
\pause
\begin{block}{Example}
$q \wedge (q \vee p)$ \\
$p$ is redundant because the formula is logically equivalent to just $q$.
\end{block}
\end{frame}

\begin{frame}
\frametitle{Prime Implicates}
\begin{itemize}
    \item A \textbf{clause} $c$ is an \textbf{implicate} of $A$ if $A \rightarrow c$.
    \item A \textbf{prime implicate} is a minimal implicate. Removing any literal would break the entailment.
    \item A \textbf{covering set} of prime implicates uniquely determines $A$.
\end{itemize}
\pause
\begin{block}{Example}
If $A = p \wedge q$, then $\{p\}$ and $\{q\}$ are prime implicates, but $\{p \vee q\}$ is not minimal.
\end{block}
\end{frame}

\begin{frame}
\frametitle{Interpretations and Distance}
\begin{itemize}
    \item An \textbf{interpretation} $w$ assigns a truth value to each atom in $\mathit{atm}$.
    \item $\llbracket A \rrbracket$ = set of interpretations where $A$ is true.
    \item Distance between two worlds $w$ and $v$:
    \[
        \text{DIST}(w, v) = \{p \in \mathit{atm} : w(p) \ne v(p)\}
    \]
\end{itemize}
\pause
\begin{block}{Example}
$w = \{p, q, \neg r\}, \quad v = \{p, \neg q, r\}$  
$\Rightarrow \text{DIST}(w, v) = \{q, r\}$
\end{block}
\end{frame}

% Intuitions

\begin{frame}
\frametitle{Agent Assumptions}
\begin{itemize}
    \item (1) The belief base represents the agent's view of the actual world, using propositional formulas.
    \item (2) The agent receives input via sensors. This input is propositional as well.
    \item (3) The input is assumed to reflect an event that really occurred in the world.
    \item (4) The agent updates the belief base while staying as close as possible to what was previously believed.
    \item (5) The agent does not know if the input is noisy or incorrect.
\end{itemize}
\pause
\begin{block}{Implication}
These assumptions shape the behavior and justification of different update operators.
\end{block}
\end{frame}

%=========UPDATE OPERATORS=========%

\begin{frame}
\frametitle{Update Operators Overview}
\begin{itemize}
    \item Belief update modifies the current world $w$ to make $A$ true with minimal change.
    \item Different operators use different definitions of “minimal.”
    \item We cover 10 operators: PMA, FORBUS, MCD, MCD*, WSS, WSS$\downarrow$, WSS$^{\text{dep}}$, WSS$\downarrow^{\text{dep}}$, MCE, and MPMA$^{\gg}$.
    \item Some operators are designed with specific environments in mind (e.g., causality, sensor noise, known variable constraints).
\end{itemize}
\end{frame}

\begin{frame}
\frametitle{PMA (Possible Models Approach)}
\begin{itemize}
    \item Idea: Keep the models of $A$ that are \textbf{closest} to $w$ using \textbf{set inclusion} on differences.
    \item Defined by:
    \[
    w \cdot_{pma} [[A]] = 
    \]
    \[
    \{\, u \in [[A]] : \forall u' \in [[A]], \text{DIST}(w, u') \not\subset \text{DIST}(w, u) \,\}
    \]
    \item Satisfies all KM postulates.
    \item Treats disjunctions as exclusive-or (XOR, $\oplus$).
    \pause
    \begin{block}{Example}
    Let $w = \{\neg p, \neg q\}$ and $A = p \lor q$.
    
    PMA returns:
    \[
    \{p, \neg q\}, \quad \{\neg p, q\}
    \]
    
    It excludes:
    \[
    \{p, q\}
    \]
    
    This shows that PMA treats $p \lor q$ as exclusive-or.
    \end{block}
\end{itemize}
\end{frame}

\begin{frame}
\frametitle{FORBUS (Numeric Minimal Change)}
\begin{itemize}
    \item Like PMA, but minimizes the \textbf{number of differing atoms} instead of subset inclusion.
    \item Prefers models with the \textbf{smallest distance cardinality}.
    \item Still satisfies all KM postulates.
    \item Also treats disjunctions as exclusive.
\end{itemize}
\pause
\[
w \cdot_{forbus} [[A]] = 
\]
\[
\{ u \in [[A]] : \forall u' \in [[A]],\ \text{card}(\text{DIST}(w, u)) \leq \text{card}(\text{DIST}(w, u')) \}
\]
\pause
\small
\begin{block}{Example}
Let $w = \{\neg p, \neg q\}$ and $A = p \lor q$.

Forbus selects:
\[
\{p, \neg q\}, \quad \{\neg p, q\}
\]

It excludes:
\[
\{p, q\} \quad \text{since } \text{card}(\text{DIST}(w, \{p, q\})) = 2
\]

The selected models have distance 1 from $w$.
\end{block}
\end{frame}

\begin{frame}
\frametitle{MCD (Cone Expansion)}
\begin{itemize}
    \item MCD starts with PMA results.
    \item Then it adds all models that contain them (supersets).
    \item This set is called the "cone" of the original models.
    \item The goal is to include nearby models that still satisfy $A$.
    \item This fixes the disjunction issue by keeping more possible outcomes.
\end{itemize}
\pause
\begin{block}{Example}
$w = \{\neg p, \neg q\},\quad A = p \lor q$

PMA returns:
\[
\{p, \neg q\}, \quad \{\neg p, q\}
\]

MCD adds:
\[
\{p, q\}
\]
\end{block}
\end{frame}

\begin{frame}
\frametitle{MCD* (Fixpoint Extension)}
\small
\begin{itemize}
    \item MCD* applies the MCD step repeatedly.
    \item It keeps expanding the set of models until no more can be added.
    \item This is called reaching a fixpoint.
    \item MCD* includes the widest possible update set while still satisfying $A$.
    \item It gives better uncertainty handling than PMA or Forbus.
\end{itemize}
\pause
\small
\begin{block}{Example}
Let $w = \{\neg p, \neg q, \neg r\}$ and $A = p \lor q$.

PMA returns:
\[
\{p\}, \quad \{q\}
\]

MCD adds:
\[
\{p, q\}, \quad \{p, r\}, \quad \{q, r\}, \quad \{p, q, r\}
\]

MCD* continues expanding:
\[
\{p, q, r, s\}, \quad \{p, q, r, t\}, \ldots
\]

It includes all supersets of these that still satisfy $A$, until no new ones can be added.
\end{block}
\end{frame}

\begin{frame}
\frametitle{WSS and WSS$\downarrow$}
\small
\begin{itemize}
    \item \textbf{WSS}: Only atoms that appear in the formula $A$ are allowed to change. All other atoms must stay fixed.
    \item This makes WSS very syntax-sensitive. Even logically useless atoms can freeze their value.
    \item \textbf{WSS$\downarrow$}: First removes redundant parts of $A$ (like tautologies), then applies WSS.
    \item This makes it more semantic and less sensitive to how $A$ is written.
    \item WSS$\downarrow$ is also called Hegner or MPMA.
\end{itemize}
\pause
\small
\begin{block}{Example}
Let $w = \{\neg p, \neg q, \neg r, \neg s\}$ and $A = (p \lor q) \land r \land (s \lor \neg s)$

\textbf{In WSS}:
All atoms that appear in $A$ ($p$, $q$, $r$, $s$) are tracked.
So $s$ is frozen, even though $(s \lor \neg s)$ is always true.

\textbf{In WSS$\downarrow$}:
The tautology $(s \lor \neg s)$ is removed.
Now $s$ is no longer considered part of the update.
So $s$ can change freely if needed.
\end{block}
\end{frame}

\begin{frame}
\frametitle{MCE (Minimal Change with Exceptions)}
\begin{itemize}
    \item Based on WSS$\downarrow$, but allows certain atoms to change when needed to avoid conflicts.
    \item Computes an exception set using the prime implicates of the belief base.
    \item These atoms can change even if they don’t appear in $A$.
    \item Allows better handling of disjunctions than PMA.
    \item Not all KM postulates are satisfied.
\end{itemize}
\end{frame}

\begin{frame}
\frametitle{MCE Example}
\begin{block}{Example}
Let $w = \{\neg p, \neg q, r\}$ and $A = p \lor q$.

WSS$\downarrow$:
Only $p$ and $q$ can change. $r$ is frozen.

MCE:
If $r$ causes a conflict with a prime implicate of the belief base, it is added to the exception set.

Now $r$ can change, making models like $\{p, \neg r\}$ or $\{q, \neg r\}$ valid.
\end{block}

\begin{block}{Why $r$ Matters Here}
\begin{itemize}
    \item Suppose the belief base $B$ implies $p \rightarrow \neg r$ (i.e., $\neg p \lor \neg r$ is a prime implicate).
    \item Updating $w$ with $A$ might give $\{p, r\}$ which violates this implicate.
    \item MCE resolves the conflict by unfreezing $r$, allowing $\{p, \neg r\}$ as a valid updated world.
\end{itemize}
\end{block}
\end{frame}

\begin{frame}
\frametitle{WSS$^{\text{dep}}$ and WSS$\downarrow^{\text{dep}}$}
\begin{itemize}
    \item These are dependency-aware versions of WSS and WSS$\downarrow$.
    \item They allow changes to atoms that \textbf{depend on} those in $A$.
    \item A dependency means: the value of one atom is logically or causally linked to another.
    \item This makes updates more flexible and realistic in structured environments.
\end{itemize}
\pause
\begin{block}{Example}
If $q$ always depends on $p$, then updating $p$ may allow $q$ to change too.
\end{block}
\end{frame}

\begin{frame}
\frametitle{MPMA$^{\gg}$ (Modified PMA with Causal Rules)}
\begin{itemize}
    \item Based on WSS$\downarrow$, but adds \textbf{causal rules} of the form: $A \Rightarrow C$.
    \item If $A$ becomes true, then $C$ must also become true.
    \item Good for modeling real-world consequences of actions (e.g., pressing a button causes light to turn on).
    \item Can enforce integrity constraints and reflect domain-specific rules.
    \item MPMA$^{\gg}$ = WSS$\downarrow$ + causality.
\end{itemize}
\pause
\begin{block}{Example}
Let $A =$ “button pressed”, and rule: $A \Rightarrow C$.

Then MPMA$^{\gg}$ ensures $C =$ “light on” is also included in the updated world.
\end{block}
\end{frame}

%=========STRENGTH COMPARISON=========%

\begin{frame}
\frametitle{Update Operator Relations}
We'll explore the key relationships between different belief update operators using lemmas and theorems from the paper "Propositional Belief Base Update".
\end{frame}


\begin{frame}
\frametitle{Lemma 16: FORBUS $\subseteq$ PMA}
\textbf{Statement:} For all new information $A$ and world $w$, $w \circ_{Forbus} A \subseteq w \circ_{PMA} A$.

\textbf{Proof Idea:}
\begin{itemize}
\item Assume $u \in w \circ_{Forbus} A$ but $u \notin w \circ_{PMA} A$
\item Then there exists $v \in w \circ_{PMA} A$ with $DIST(w, v) < DIST(w, u)$
\item This contradicts the FORBUS definition (selects \textit{closest} models)
\item Hence, any model accepted by FORBUS must also be accepted by PMA
\end{itemize}
\end{frame}


\begin{frame}
\frametitle{Lemma 18: MCD ⊆ MCD*}
\textbf{Statement:} $w \circ_{MCD} A \subseteq w \circ_{MCD^*} A$

\textbf{Key Idea:}
\begin{itemize}
\item MCD*: recursively explores models when a cone contains \textgreater 1 minimal model
\item It goes deeper: builds cones over new models and re-applies PMA
\item So MCD* may uncover additional justified models not found by MCD
\end{itemize}
\end{frame}

\begin{frame}
\frametitle{Lemma 19: MCE and MCD Incomparable}
\textbf{Statement:} There exist $w, A$ such that:
\begin{itemize}
\item $w \circ_{MCE} A \not\subseteq w \circ_{MCD} A$
\item $w \circ_{MCD} A \not\subseteq w \circ_{MCE} A$
\end{itemize}

% \textbf{Example:}
% \begin{itemize}
% \item $w = {\neg p, \neg q, \neg r}$
% \item $A$ has models: $v_1 = {\neg p, q, \neg r}$, $v_2 = {p, \neg q, \neg r}$, $v_3 = {p, q, r}$
% \item Prime implicate failed: $p \lor q$ $\Rightarrow EXC(A) = {p, q}$
% \item $v_3$ changes $r \notin EXC(A)$ $\Rightarrow v_3 \notin MCE(w, A)$
% \item But $v_3$ is logically supported via cones $\Rightarrow v_3 \in MCD(w, A)$
% \end{itemize}
\end{frame}

\begin{frame}
\frametitle{Lemma 20: Literal in Prime Implicates}
\textbf{Statement:} If literal $L$ appears in a prime implicate of $A$, then some model of $A$ satisfies $L$.

\textbf{Proof Idea:}
\begin{itemize}
\item Assume $L \lor c$ is a prime implicate, but $L$ is false in all models
\item Then $A \Rightarrow \neg L$ $\Rightarrow A \Rightarrow c$
\item So $L \lor c$ not minimal; $c$ alone suffices $\Rightarrow$ contradiction
\end{itemize}
\end{frame}

\begin{frame}
\frametitle{Theorem 21: Only Clauses False in $w$ Matter}
\textbf{Statement:} $w \circ_{PMA}(C1 \cup C2) = w \circ_{PMA}(C2)$ where $C1$ = clauses true in $w$, $C2$ = false ones.

\textbf{Explanation:}
\begin{itemize}
\item PMA minimizes changes needed to satisfy input
\item If clause in $C1$ is already true, it doesn’t influence model selection
\item Proof uses induction: show adding one satisfied clause doesn't change result
\end{itemize}
\end{frame}

\begin{frame}
\frametitle{Theorem 22: Difference Atoms = Exceptional Atoms}
\textbf{Statement:} $\bigcup_{v \in w \circ_{PMA} A} DIST(w, v) = atm(C2) = EXC(A)$

\textbf{Interpretation:}
\begin{itemize}
\item The atoms that change in PMA models are exactly those in falsified prime implicates
\end{itemize}
\end{frame}

\begin{frame}
\frametitle{Theorem 23: PMA ⊆ MCE}
\textbf{Statement:} $w \circ_{PMA} A \subseteq w \circ_{MCE} A$, but not equal

\textbf{Explanation:}
\begin{itemize}
\item PMA: closest models to $w$
\item MCE: any model differing only on exceptional atoms
% \item Example 13 shows MCE can include more than PMA
\end{itemize}
\end{frame}

\begin{frame}
\frametitle{Theorem 24: MCE ⊆ WSS}
\textbf{Statement:} $w \circ_{MCE} A \subseteq w \circ_{WSS} A$, and sometimes strictly

\textbf{Explanation:}
\begin{itemize}
\item MCE: respects exceptions (EXC(A))
\item WSS: allows changes on any atom in $atm(A)$
\item WSS may include models that MCE rejects
\end{itemize}
\end{frame}

%=========KM POSTULATES=========%

\begin{frame}
\frametitle{The KM Postulates}
\begin{itemize}
    \item Katsuno and Mendelzon proposed 8 rationality postulates for belief update.
    \item PMA and FORBUS satisfy all 8. Others satisfy only some.
    \item The postulates define how updates should behave logically and semantically.
\end{itemize}
\pause
\begin{block}{Notation}
$B \diamond A$ = update of belief base $B$ with formula $A$
\end{block}
\end{frame}

\begin{frame}
\frametitle{Core KM Postulates (U1–U4)}
\begin{itemize}
    \item \textbf{(U1)} After update, the new info must hold. \\
    $\quad B \diamond A \rightarrow A$
    
    \item \textbf{(U2)} If $B$ already implies $A$, then updating changes nothing. \\
    $\quad B \rightarrow A \Rightarrow B \diamond A \equiv B$
    
    \item \textbf{(U3)} If $B$ and $A$ are consistent, so is the update. \\
    $\quad B \diamond A$ is consistent if $B$ and $A$ are.
    
    \item \textbf{(U4)} Equivalent formulas update the same way. \\
    $\quad A_1 \equiv A_2 \Rightarrow B \diamond A_1 \equiv B \diamond A_2$
\end{itemize}
\end{frame}

\begin{frame}
\frametitle{Advanced KM Postulates (U5–U6)}
\begin{itemize}
    \item \textbf{(U5)} Updating with $A \wedge C$ should include $C$ in the result. \\
    $\quad B \diamond (A \wedge C) \rightarrow (B \diamond A) \wedge C$
    
    \item \textbf{(U6)} If two inputs imply each other after update, treat them the same. \\
    $\quad B \diamond A_1 \rightarrow A_2$ and $B \diamond A_2 \rightarrow A_1$ \\
    $\Rightarrow B \diamond A_1 \equiv B \diamond A_2$
    
    \item \textbf{(U7–U8)} Less common chaining rules, not covered in this paper.
\end{itemize}
\end{frame}

\begin{frame}
\frametitle{Operator Postulate Satisfaction}
\begin{itemize}
    \item \textbf{PMA, Forbus:} satisfy all postulates. Very strict, minimal change.
    \item \textbf{MCD, MCD*, MCE:} violate (U2) and (U5) to support disjunctions.
    \item \textbf{WSS:} violates (U2), (U4), and others due to syntax sensitivity.
    \item \textbf{WSS$\downarrow$, MPMA$^{\gg}$:} more semantic than WSS, but still violate some postulates.
    \item \textbf{WSS$^{\text{dep}}$, WSS$\downarrow^{\text{dep}}$:} improve flexibility via dependencies, but still not fully compliant.
\end{itemize}
\end{frame}

\begin{frame}
\frametitle{Violations and Justifications}
\begin{itemize}
    \item \textbf{(U2):} Violated when $A$ is already implied by $B$ but introduces ambiguity (e.g., $p \lor q$).
    \item \textbf{(U5):} Leads to exclusive-or behavior. Only one disjunct survives, even if both are possible.
\end{itemize}
\pause
\begin{block}{Example}
$B = \neg p \wedge \neg q$, $A = p \lor q$ \\
PMA returns only $\{p, \neg q\}$ and $\{\neg p, q\}$ \\
Excludes $\{p, q\}$ even though it satisfies $A$
\end{block}
\begin{itemize}
    \item This reflects a mismatch between strict logic and real-world reasoning.
    \item For example, sensors might detect “someone moved” but not say who, then the input is $p \lor q$.
    \item Allowing both $\{p\}$ and $\{q\}$ and even $\{p, q\}$ is more reasonable.
\end{itemize}
\end{frame}

%=========INTEGRITY CONSTRAINTS=========%
\begin{frame}
\frametitle{Handling Integrity Constraints}
\textbf{Goal:} Ensure belief update operations respect constraints that must always hold, regardless of new information.

\textbf{Example:}
\begin{itemize}
\item IC: $Light \Leftrightarrow (Up1 \Leftrightarrow Up2)$
\item This must remain true in all updated belief states
\end{itemize}
\end{frame}

\begin{frame}
\frametitle{The Problem with Conjoining IC to Input}
\textbf{Old approach:} $B \circ_{IC} A = B \circ (A \land IC)$

\textbf{Issue:}
\begin{itemize}
\item Operators like PMA, MCD, MCE, WSS mishandle this
\item Example:
\begin{itemize}
\item $B = Up1 \land Up2 \land Light$, update with $\neg Up1$
\item Expect: only $Up1$ changes, $Up2$ stays, $Light$ turns off
\item Actual: operators may change $Up2$ or break the IC
\end{itemize}
\end{itemize}
\end{frame}

\begin{frame}
\frametitle{$MPMA^{\gg}$: Causal PMA Update}
\textbf{Solution:} Separate causal rules and apply them after minimal change.

\textbf{Update Process:}
\begin{itemize}
\item Perform minimal change update with $A$ only
\item Enforce IC and causal rules after update
\end{itemize}

\textbf{Formal:}



\begin{itemize}
\item $CR$: Causal rules (e.g., $Up1 \Leftrightarrow Up2 \Rightarrow Light$)
\item $T(CR)$: Translation to implications
\end{itemize}
\end{frame}

\begin{frame}
\frametitle{$WSS_{dep}$: Controlled Atom Updates}
\textbf{Idea:} Only allow changes to atoms that depend on the input.

\textbf{Steps:}
\begin{itemize}
\item Define $dep(p)$: atoms that can change when $p$ changes
\item $dep(A) = \bigcup dep(p)$ for all atoms $p$ in $A$
\item Restrict $DIST(w, u) \subseteq dep(A)$
\end{itemize}
\end{frame}


\begin{frame}
\frametitle{$WSS_{dep}$ Example}
\textbf{Let:}
\begin{itemize}
\item $dep(Up1) = {Up1, Light}$
\item $dep(Up2) = {Up2, Light}$
\item $dep(Light) = {Up1, Up2, Light}$
\end{itemize}

Update: $w = {Up1, Up2, Light}$ with $\neg Up1$

\textbf{Allowed Models:}
\begin{itemize}
\item ${\neg Up1, Up2, \neg Light}$ \textcolor{green}{✓ valid}
\item ${\neg Up1, \neg Up2, Light}$ \textcolor{red}{✗ invalid (Up2 changed)}
\end{itemize}

\textbf{Final step:} Filter models that violate IC
\end{frame}

%=========OTHER APPROACHES=========%
\begin{frame}
\frametitle{Other Approaches to Belief Base Update}
\begin{itemize}
\item \textbf{Syntax-Based Methods:} Transform the belief base via formula manipulation rather than models.
\item \textbf{Relevance/Dependence-Based:} Limit updates to dependent atoms (e.g., $WSS_{dep}$).
\item \textbf{Causal Reasoning:} Use causal rules or action-based frameworks (e.g., $MPMA^{\gg}$).
\item \textbf{Belief Merging/Revision:} Merge multiple sources or perform revision under different logic.
\end{itemize}
\end{frame}

%=========CONCLUSION=========%

\begin{frame}
\frametitle{Conclusion}
\begin{itemize}
\item Belief update aims to integrate new information with minimal, rational change.
\item Different operators (PMA, MCD, MCE, etc.) reflect different interpretations of "minimal change".
\item KM postulates guide the evaluation but are not always satisfied.
\item Handling integrity constraints (ICs) often requires causal-aware methods like $MPMA^{\gg}$.
\item No one-size-fits-all: operators should match the needs of the application domain.
\end{itemize}
\end{frame}


% End

\begin{frame}[c]
\centering
\Huge
\textbf{Thank you!}

\vspace{1cm}

\LARGE
Questions?
\end{frame}

\end{document}